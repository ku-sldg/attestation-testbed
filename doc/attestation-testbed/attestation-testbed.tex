\documentclass[runningheads]{llncs}
%
\usepackage{graphicx}
% Used for displaying a sample figure. If possible, figure files should
% be included in EPS format.
\usepackage{tikz}
\usetikzlibrary{arrows}
\usepackage{verbatim}
%\usepackage{amsmath}
%\usepackage{amssymb}
%\usepackage{graphicx}
%\usepackage[all]{xy}
\usepackage{array}
\usepackage{enumitem}
%\usepackage{cite}
\usepackage{natbib}
% If you use the hyperref package, please uncomment the following line
% to display URLs in blue roman font according to Springer's eBook style:
% \renewcommand\UrlFont{\color{blue}\rmfamily}
\usepackage[breaklinks=true]{hyperref}
\usepackage{breakcites}
\renewcommand\UrlFont{\color{blue}\rmfamily}

\begin{document}
%
\title{Contribution Title\thanks{Supported by organization x.}}
%
%\titlerunning{Abbreviated paper title}
% If the paper title is too long for the running head, you can set
% an abbreviated paper title here
%
\author{First Author \and
Second Author \and
Third Author}
%
\authorrunning{F. Author et al.}
% First names are abbreviated in the running head.
% If there are more than two authors, 'et al.' is used.
%
\institute{Information and Telecommunication Technology Center \\ The
  University of Kansas \\ Lawrence, KS 66045 \\
  \email{\{authors\}@ku.edu}}
%
\maketitle              % typeset the header of the contribution
%
\begin{abstract}
The abstract should briefly summarize the contents of the paper in
15--250 words.

\keywords{First keyword  \and Second keyword \and Another keyword.}
\end{abstract}

\section{Introduction}
\label{sec:introduction}

\begin{figure}[hbtp]
  \centering

  \tikzstyle{block}=[draw,rectangle,minimum size=0.75cm]
  
  \begin{tikzpicture}
    \node[block] (HR0) at (-2,1){$H_0$};
    \node[block] (HR1) at (-1,1){$H_1$};
    \node[block] (HR2) at (0,1){$H_2$};
    \node[block] (HR3) at (1,1){$H_3$};
    \node[block] (HR4) at (2,1){$H_4$};
    \node (TPMB) at (2,-1){$(B,\{B^{-1}\}_{K})$};
    \node(TPMA) at (-2,-1){$(A,\{A^{-1}\}_{K})$};
    \node[block] (B) at (2,-3){$B$};
    \node[block] (A) at (-2,-3){$A$};

    \node[color=blue] at (2,-4) (BINI) {$\mathsf{B.INI}$};
    \node[color=red] at (-2,-4) (AINI) {$\mathsf{A.INI}$};

    \node[color=green] at (-0,-5) (AINI) {$\mathsf{MANIFEST}$};
    
    \draw[-stealth,color=red] (A.north) -- (TPMA.south);
    \draw[-stealth,color=blue] (B.north) -- (TPMB.south);

    \draw[-stealth,color=red] (TPMA.north) -- (HR0.south);
    \draw[-stealth,color=blue] (TPMB.north) -- (HR1.south);
    \draw[-stealth,color=red] (TPMA.north) -- (HR2.south);
    \draw[-stealth,color=red] (TPMA.north) -- (HR3.south);
    \draw[-stealth,color=blue] (TPMB.north) -- (HR4.south);
        
    \draw[-stealth,color=green] (A.east) -- (B.west);

    \node[color=black] (BC) at (4,1){\textsf{geth}};
    \node[color=black] (KS) at (4,-1){\textsf{keys}};
    \node[color=black] (AM) at (4,-3){\textsf{AMs}};
    \node[color=black] (AM) at (4,-4.5){\textsf{configs}};
  \end{tikzpicture}
  
  \caption{Big Picture}
  \label{fig:big-picture}
\end{figure}

%
%
%
% ---- Bibliography ----
%
% BibTeX users should specify bibliography style 'splncs04'.
% References will then be sorted and formatted in the correct style.
%
%\bibliographystyle{splncs04}
\bibliographystyle{splncsnat}
\bibliography{sldg}
%
\end{document}
